\documentclass[a5paper,9pt]{scrartcl}
\usepackage[utf8]{inputenc}
\usepackage{menukeys}
\usepackage{listings}
\usepackage{hyperref}


\newcommand*{\mybox}[1]{\framebox{#1}}

\title{Windows workflow}

\begin{document} 
    \section{Johdanto}
    Tässä dokumentissa käydään läpi erilaisia tapoja helpottaa ja nopeuttaa tutkijan työskentelyä. Aluksi käymme läpi yksinkertaisia, tietokoneen sisään rakennettuja ominaisuuksia, jonka jälkeen siirrymme monimutkaisempiin ongelmiin.
    
    \section{Peruspikanäppäimet}
    Lämmitellään ensin peruspikanäppäimillä, jotka toimivat ilman kaikilla windows-koneilla, ilman Autohotkeyta.
    
    \subsection{Windows}
    
    \keys{Ctrl}+\keys{c} Kopioi 
    
    \keys{Ctrl}+\keys{v} Liitä 
    
    \keys{Ctrl}+\keys{x} Leikkaa 
    
    \keys{Ctrl}+\keys{z} Kumoa
    
    \keys{Ctrl}+\keys{Shift}+\keys{z} tai \keys{Ctrl}+\keys{y} Vastatoiminto kumoamiselle (vaihtelee ohjelmittain)
    
    \keys{Alt}+\keys{Tab} Palaa edelliseen avattuun ikkunaan
    
    \subsection{Firefox/Chrome}
    
    \keys{Ctrl}+\keys{Tab} mene seuraavaan välilehteen
    
    %\keys{Ctrl}+\keys{1} mene 1. välilehteen (toimii vastaavasti muilla numeroilla)
	\url{https://support.mozilla.org/en-US/kb/keyboard-shortcuts-perform-firefox-tasks-quickly}
	
	
	%testing:
	%\expandafter\def\expandafter\UrlBreaks\expandafter{https://support.mozilla.org/en-US/kb/keyboard-shortcuts%
	%	-perform-firefox-tasks-quickly}
	
	
	
    
    \url{https://support.google.com/chrome/answer/157179?hl=en}
    
    
	\subsection{Outlook}
	
	\url{https://support.office.com/en-us/article/keyboard-shortcuts-for-outlook-3cdeb221-7ae5-4c1d-8c1d-9e63216c1efd}
    
    \section{Ladattavat ohjelmat ja koodi}
    Osiota 2 lukuunottamatta tarvitset seuraavat ohjelmat ja koodin.
    
    \subsection{Zotero}
    \url{https://www.zotero.org/download/}
    
    
    Lataa sekä ''Zotero for Windows'', että ''Zotero Connector''. Tämän jälkeen viitteet saat tallennettua Firefoxissa/Chromessa painamalla hiirellä selaimen oikeaa yläkulmaa (edellyttäen, että Zotero-sovellus on auki).
    
    Tallenna tiedosto \textbf{vtl\_workflow.ahk} haluamaasi kansioon koneellasi. Kun klikkaat tiedosta, se aktivioituu. Jotta scripti pyörii myös sen jälkeen, kun olet käynnistänyt tietokoneen uudestaan, luo tiedostosta pikakuvake ja laita se windowsin käynnistys-kansioon.
    
    \emph{WINDOWS 10}: Paina \keys{Win}+\keys{r} ja kirjoita ilmestyvään ruutuun \mybox{shell::startup} ja paina \keys{Enter}. Tämän jälkeen Copy-pastaa luomasi pikakuvake aloituskansioon
    
	\emph{WINDOWS 7}: Paina \keys{Win}+\keys{r} ja kirjoita ilmestyvään ruutuun 
	
	\mybox{C:/users/\textbf{*YOUR USERNAME*}/AppData/Roaming/Microsoft/Windows/Start Menu/Programs/Startup} ja paina \keys{Enter}. *YOUR USERNAME* korvataan siis sinun käyttäjätunnuksellasi, esim. mmak (sama millä kirjaudutaan windowsiin ja Outlook webiin)
	Tämän jälkeen Copy-pastaa luomasi pikakuvake aloituskansioon.
    
    \subsection{autohotkey}
    \url{https://www.autohotkey.com/download/}
    
    Valitse ''Download Autohotkey Installer''
    \subsection{Word}
    Word on luultavasti sinulla  jo valmiiksi asennettuna, mutta tarvitset koodinpätkän, jotta seuraavat jutut toimivat. Kun word on auki paina \keys{Alt}+\keys{F11} ja Copy-pastaa (\keys{Ctrl}+\keys{c} \& \keys{Ctrl}+\keys{v}) tiedostossa \textbf{word\_macros.txt} oleva koodi käytettävissä olevaan isoon tilaan, tallenna (\keys{Ctrl}+\keys{s}) ja käynnistä word uudelleen.
    %\begin{lstlisting}[frame=single]
    %\end{lstlisting}
    
    \subsection{Stata}
    Stata15 lisäksi tarvitset tiedoston \textbf{stata\_workflow.do} testausta varten. Eri pikanäppäimet tosin toimivat ilman tätä tekstitiedostoa

    
    \section{Viitteet}
    \subsection{Lähdeviitteiden tarkistaminen Wordissa}
    Käytettyjen lähdeviitteiden tarkistaminen onnistuu kätevästi jakamalla ruutu kahtia.
    
    \menu{Näytä > (Järjestä) > Jaa }
    
    \menu{View > (Arrange) > New Window}
    
    Ks. lisätiedot \url{https://support.office.com/en-us/article/view-two-parts-of-a-document-at-the-same-time-in-word-for-mac-1adf3317-0ec4-4568-ad32-6f68b3e4b386}
    
    Tämän jälkeen lue teksti alusta loppuun (1. osa dokumenttia) ja merkitse käytetyt viitteet korostamalla ne. Yhden lähdeviitteen (kappaleen) korostaminen onnistuu pikakomennolla
    
    \keys{Ctrl}+\keys{ö}. 
    
    Jos haluat poistaa värityksen yhdestä viitteestä (kappaleesta) paina
    
     \keys{Ctrl}+\keys{ä}
     
    Kun olet päässyt tekstin loppuun, voit poistaa värjäämättömät viitteet, poistaa korostuksen ja dokumentin kahtia jaon.
    
    \subsection{Lähdeviittaaminen Wordissa}
    
    \keys{Ctrl}+\keys{å}
    Tämä pikakomento tekee monta asiaa, riippuen tilanteesta:
    
    - lisää uuden viitteen
    
    - muokkaa olemassa olevaa viitettä
    
    - poistaa viittestä kirjoittajan
    
    - lisää viitteeseen kirjoittajan (edellisen käänteistoiminto)
    
    - avaa viittausikkunan, jos se on olemassa, mutta ei ole aktiivinen
    
    
    Lisäksi on hyvä tietää, että kaksoispisteen \keys{:} avulla viitteen perään saa sivunumerot.
    
    Kun dokumentti on valmis, lisää lähdeluettelo wordiin pikanäppäimellä:
    
    \keys{Ctrl}+\keys{Shift}+\keys{å}
    
    Näin välttyy kokonaan viitteiden tarkistamiselta, ja monelta muulta harmilta.
    
    \section{Stata}
    
    
    \section{Mitä muuta?}
    Youtube, eteenpäin ja nopeus
    
    \section{Extra}
    You can visualize paths \directory{/home/moose/Desktop/manual.tex}
    or menus \menu{View > Highlight Mode > Markup > LaTeX} or key
    press combinations: \keys{\ctrl + \shift + F} is for formatting
    in Eclipse.
    You can also visualize \keys{\tab}, \keys{\capslock}, \keys{\Space}, 
    \keys{\arrowkeyup} and many more.
    Let's try some more: \keys{a}, \keys{ä} 

\end{document}