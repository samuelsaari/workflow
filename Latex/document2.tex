\documentclass[a5paper,9pt]{scrartcl}
\usepackage[utf8]{inputenc}
\usepackage{menukeys}
\usepackage{listings}

\title{Windows workflow}

\begin{document}
	\section{Pikanäppäimiä}
	You can visualize paths \directory{/home/moose/Desktop/manual.tex}
    or menus \menu{View > Highlight Mode > Markup > LaTeX} or key
    press combinations: \keys{\ctrl + \shift + F} is for formatting
    in Eclipse.
    You can also visualize \keys{\tab}, \keys{\capslock}, \keys{\Space}, 
    \keys{\arrowkeyup} and many more.
    Let's try some more: \keys{a}, \keys{ä}  
    \section{Johdanto}
    Tässä dokumentissa käydään läpi erilaisia tapoja helpottaa ja nopeuttaa tutkijan työskentelyä. Aluksi käymme läpi yksinkertaisia, tietokoneen sisään rakennettuja ominaisuuksia, jonka jälkeen siirrymme monimutkaisempiin ongelmiin.
    
    \section{Ladattavat ohjelmat ja koodi}
    Osiota 1 lukuunottamatta tarvit seuraavat ohjelmat ja koodin.
    
    \subsection{Zotero}
    zotero.org
    
    \subsection{autohotkey}
    autohotkey.com
    
    \subsection{Word}
    Word on kaikilla jo valmiiksi asennettuna. Kun word on auki paina \keys{Alt}+\keys{F11} ja Copy-pastaa (\keys{Ctrl}+\keys{c} \& \keys{Ctrl}+\keys{v}) tiedostossa wordmacros.txt oleva koodi käytettävissä olevaan isoon tilaan.
    %\begin{lstlisting}[frame=single]
    %\end{lstlisting}
    
    \section{Peruspikanäppäimet}

\end{document}